
\begin{itemize}[leftmargin=1cm]
    \item $\mu$-TEL --- goal-conditioned memory,
    \item $\Lambda$ --- structural transformation,
    \item $\Omega^*$ --- orientational shaping,
    \item $\circledcirc$ --- unitary stabilization,
    \item $C\Omega$ --- coherence field,
    \item PTOr --- phase-transition operator.
\end{itemize}

Higher-level modules of the full SV architecture—such as hierarchical governance, multi-agent teleology, and ecological embedding—are deliberately out of scope.

The purpose of this document is therefore two-fold:

\begin{enumerate}[leftmargin=1cm]
    \item to present a minimal operational prototype that researchers can directly test,
    \item to establish a scientific foundation for forthcoming SV extensions.
\end{enumerate}

\section{Architecture Overview}

\subsection{The Teleological Pipeline}

The full pipeline is:

\[
\Phi^* \rightarrow \mu\text{-TEL} \rightarrow \Lambda \rightarrow \Omega^* \rightarrow \circledcirc \rightarrow C\Omega \rightarrow \text{PTOr}
\]

Each operator corresponds to a distinct cognitive function.

\begin{itemize}[leftmargin=1cm]
    \item \textbf{$\Phi^*$ — Presence}: initial orientational encoding.
    \item \textbf{$\mu$-TEL — Teleological Memory}: gated memory updated toward a goal vector.
    \item \textbf{$\Lambda$ — Structure}: extracts and organizes semantic features.
    \item \textbf{$\Omega^*$ — Orientation}: produces directional activation.
    \item \textbf{$\circledcirc$ — Unitary Core}: normalizes the state into a stable manifold.
    \item \textbf{$C\Omega$ — Coherence Field}: smooth self-correction across cycles.
    \item \textbf{PTOr — Phase Transition Operator}: creates a qualitative change producing a decision-ready vector.
\end{itemize}

\subsection{Design Principles}

SV-Core follows four principles:

\begin{enumerate}[leftmargin=1cm]
    \item teleology precedes action,
    \item no retraining required,
    \item compositional modularity,
    \item compatibility with modern LLMs.
\end{enumerate}

\section{Implementation Summary}

SV-Core is implemented in approximately 200 lines of PyTorch:

\begin{itemize}[leftmargin=1cm]
    \item no custom CUDA kernels,
    \item no model retraining,
    \item fully differentiable,
    \item memory registered via \texttt{register\_buffer}.
\end{itemize}

It includes a class per operator and an \texttt{SVCore} controller orchestrating the pipeline.

\section{Integration with LLMs}

\subsection{Hidden-State Extraction}

Any transformer exposes hidden states. SV-Core uses the last-token embedding as the initial state $\Phi^*$.

\subsection{Fusion Strategies}

A minimal integration uses:

\[
\text{fused} = h_{\text{LLM}} + h_{\text{SV}}
\]

Other approaches (attention biasing, planning layers) are possible.

\subsection{Agentic Behavior}

SV-Core provides:

\begin{itemize}[leftmargin=1cm]
    \item persistent direction,
    \item reduced drift,
    \item more stable multi-step reasoning,
    \item improved goal-conditioning.
\end{itemize}

\section{Scientific Motivation}

LLMs show local competence without global control.  
SV-Core implements:

\begin{itemize}[leftmargin=1cm]
    \item goal-aligned memory,
    \item structural filtering,
    \item manifold stabilization,
    \item predictable phase transitions.
\end{itemize}

These ideas draw from dynamical systems, cognitive architectures, and attractor dynamics.

\section{Experimental Expectations}

Predicted behaviors include:

\begin{enumerate}[leftmargin=1cm]
    \item fewer contradictions,
    \item more stable reasoning chains,
    \item inspectable internal states,
    \item reduced noise propagation,
    \item identifiable phase-transition clusters.
\end{enumerate}

\section{Limitations}

SV-Core is a minimal foundation with known constraints:

\begin{enumerate}[leftmargin=1cm]
    \item no learned teleology,
    \item no large-scale experimental validation,
    \item requires hidden-state access (not possible for closed APIs),
    \item memory dimension must match LLM hidden size,
    \item no meta-learning or reinforcement loop,
    \item no multi-agent dynamics.
\end{enumerate}

\section{Future Work}

SV-Core represents only the foundational layer of the Living System architecture.  
Advanced components—hierarchical governance, multi-scale coherence, and extended agent dynamics—will be presented in future technical reports.

\section{Conclusion}

SV-Core introduces a minimal teleological architecture compatible with real LLMs.  
It is modular, stable, reproducible, and provides a scientific basis for developing future agentic systems.

\section*{Citation}

Vinas, A. (2025). \textit{SV-Core: A Teleological Cognitive Architecture for Agentic Large Language Models}. Zenodo. \url{https://doi.org/10.5281/zenodo.17889413}

\section*{Code Availability}

GitHub: \url{https://github.com/alexjoseph-creator/sv-core}
Zenodo: \url{https://doi.org/10.5281/zenodo.17889413}

\end{document}
